\section*{ЗАКЛЮЧЕНИЕ}
\addcontentsline{toc}{section}{ЗАКЛЮЧЕНИЕ}

Был разработан программный комплекс, состоящий из дух модулей: модуль модели UNet и модуль пользовательского приложения. Приведены результаты обучения модели и показаны примеры использования разработанного программного комплекса.

Проведено исследование качества созданного программного комплекса. В результате исследования были выделены достоинства, среди которых универсальность и высокая точность, и недостатки, среди которых ошибки классификации при изменении размера изображения.

Таким образом, цель работы --- разработать метод распознавания челюстно--лицевых костей черепа по томографическим снимкам головы человека, была достигнута.