\section*{ЗАКЛЮЧЕНИЕ}
\addcontentsline{toc}{section}{ЗАКЛЮЧЕНИЕ}

Были рассмотрены задачи распознавания образов и нахождения обхектов на изображении. Были рассмотрены виды классификаторов и их применимость при распознавании костей томографическим снимкам. Рассмотрены задачи семантической сегментации и сегментации экземпляров, их трактовка в задаче распознавания челюстно--лицевых костей.

Было дано определение понятия нейронной сети, рассмотрены виды нейронных сетей и принцип их работы.

Были рассмотрены особенности применения нейронных сетей в качестве классификаторов для распознавания и классификации костей по томографическим снимкам.

Были рассмотрены и проанализированы технологии (R--CNN, Fast R--CNN и Faster R--CNN) для распознавания образов при помощи нейронных сетей, приведены преимущества и недостатки рассмотренных технологий в задаче распознавания костей по томографическим снимкам.

Были рассмотрены и проанализированы технологии (Mask R--CNN, UNet) для выделения объектов на изображении в рамках задачи работы.

Были описаны критерии выбора данных для обучения модели, послужившей основанием для разработанного метода.

Был спроектирован метод распознавания а также программный комплекс, реализующий интерфейс для метода. Выбраны данные для обучения модели, обоснован их выбор.

Был разработан программный комплекс, состоящий из дух модулей: модуль модели UNet и модуль пользовательского приложения. Приведены результаты обучения модели и показаны примеры использования разработанного программного комплекса.

Проведено исследование качества созданного программного комплекса. В результате исследования были выделены достоинства, среди которых универсальность и высокая точность, и недостатки, среди которых ошибки классификации при изменении размера изображения.

Таким образом, цель работы --- разработать метод распознавания чулюстно--лицевых костей черепа по томографическим снимкам головы человека, была достигнута.