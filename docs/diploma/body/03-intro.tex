\section*{ВВЕДЕНИЕ}
\addcontentsline{toc}{section}{ВВЕДЕНИЕ}

Согласно исследованиям \cite{facialfracs0}\cite{isaps}, на момент 2018 года 10\% всех несчастных случаев и обращений в отделения неотложной помощи приходятся на травмы челюстно--лицевой области головы человека, а 1\% всех пластических операций --- на контурную пластику (изменение формы) лица.

Для лечения лицевых поверждений требуется диагностика типа травмы, а также причины ее появления. Для диагностики обычно используются КТ и МРТ снимки головы. На основе полученных снимков можно выбрать план лечения.

Нередко при лечении челюстно--лицевых повреждений требуется хирургическое вмешательство, а проведение безоперационной контурной пластики дает непостоянный результат, требующий повторной процедуры спустя некоторое время. Поэтому пациенты часто прибегают к операции, чтобы добиться долгосрочных изменений \cite{isaps}.

Проведение операций по лечению дефектов челюстно--лицевого сустава может быть затруднительным в виду недостаточных данных, полученных при сборе анамнеза \cite{facialfracs1}. Для уточнения и расширения показаний пациента можно воспользоваться компьютерными технологиями для распознавания челюстно--лицевых костей и их повреждений. Использование нейронных сетей глубокого обучения позволяет повысить эффективность определения травм лица, что в свою очередь снижает риск замедления лечения и выбора неправильного подхода к реабилитации \cite{facialfracs2}.

Компьютерная диагностика также позволяет выявить дефекты в строении зубочелюстной системы или предотвратить развитие патологических отклонений. Кроме того, она может использоваться как вспомогательный инструмент при проектировании зубочелюстных протезов.

Цель работы --- разработать метод распознавания челюстно--лицевых костей черепа по томографическим снимкам головы человека.

Для достижения поставленной цели потребуется:
\begin{itemize}
\item Описать термины предметной области и обозначить проблему;
\item Описать технологии, с помощью которых можно реализовать метод распознавания челюстно--лицевых костей;
\item Разработать метод распознавания челюстно--лицевых костей черепа по томографическим снимкам головы человека.
\item Разработать программный комплекс, реализующий интерфейс для взаимодействия с разработанным методом.
\item Исследовать разработанный метод на применимость при работе с различными типами томографических снимков и при работе с различными проекциями одного снимка.
\end{itemize}