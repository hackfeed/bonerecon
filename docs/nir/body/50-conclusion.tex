\section*{ЗАКЛЮЧЕНИЕ}
\addcontentsline{toc}{section}{ЗАКЛЮЧЕНИЕ}

Была рассмотрена задача классификации, виды классификаторов и их применимость при распознавании костей по КТ и МРТ снимкам.

Было дано определение понятия нейронной сети, рассмотрены виды нейронных сетей и принцип их работы.

Были рассмотрены особенности применения нейронных сетей в качестве классификаторов для распознавания костей по КТ и МРТ снимкам.

Были рассмотрены и проанализированы технологии (R--CNN, Fast R--CNN и Faster R--CNN) для распознавания образов при помощи нейронных сетей, приведены преимущества и недостатки рассмотренных технологий в задаче распознавания костей по КТ и МРТ снимкам.

Таким образом, цель работы --- описать существующие методы распознавания челюстно--лицевых костей по КТ или МРТ снимкам головы человека, была достигнута.