\section*{ОПРЕДЕЛЕНИЯ, ОБОЗНАЧЕНИЯ И СОКРАЩЕНИЯ}

\begin{enumerate}[leftmargin=1.6\parindent]
\item КТ (компьютерная томография) --- ме­тод не­раз­ру­шаю­ще­го по­слой­но­го ис­сле­до­ва­ния внут­рен­ней струк­ту­ры объ­ек­та по­сред­ст­вом ска­ни­рую­ще­го про­све­чи­ва­ния его рент­ге­нов­ским из­лу­че­ни­ем \cite{kt};
\item МРТ (магнитно--резонансная томография) --- способ получения томографических медицинских изображений для исследования внутренних органов и тканей с использованием явления ядерного магнитного резонанса \cite{mrt}.
\item MLP (от англ. Multilayer perceptron) --- многослойный персептрон.
\item CNN (от англ. Convolutional neural network) --- сверточная нейронная сеть. 
\item RNN (от англ. Recurrent neural network) --- рекуррентная нейронная сеть. 
\end{enumerate}